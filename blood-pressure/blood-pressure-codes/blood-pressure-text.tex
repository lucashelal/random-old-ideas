\documentclass[a4paper,12pt]{article}
\usepackage[top=2cm,bottom=2cm,left=2cm,down=2cm,a4]{geometry}
\usepackage{amsfonts}
\usepackage{amsmath}
\usepackage{amssymb}
\usepackage{tabularx}
\usepackage{multicol}
\usepackage{multirow}
\usepackage{physics}
\usepackage{siunitx}
\usepackage{bm}
\usepackage{booktabs}
\usepackage{enumitem}
\usepackage{longtable}
\usepackage{array}
\usepackage{bm}
\usepackage{tikz}
\usepackage{pgfplots}
\usepackage{indentfirst}
\usepackage[outline]{contour} % glow around text
\usetikzlibrary{calc}
\usetikzlibrary{angles,quotes} % for pic
\usetikzlibrary{arrows.meta}
\usetikzlibrary{patterns}
\tikzset{>=latex} % for LaTeX arrow head
\contourlength{1.35pt}

\newlist{itemize}{itemize}{1}
\setlist[itemize]{
    itemindent=-0.6cm,
    label=$\circ$,
}

\newlist{enumerate}{enumerate}{1}
\setlist[enumerate]{
    itemindent=-0.4cm,
    label=$\rightarrow$,
}

\setlength{\parindent}{0cm}

\begin{document}

\section*{Sangue e Tecido Conjuntivo}

\subsection*{Propriedades do Sangue Arterial}

\subsubsection*{Composição}

\begin{enumerate}
    \item 45 \% elementos figurados (hemácias, leucócitos e plaquetas)
    \item 55 \% plasma (matriz extracelular)
\end{enumerate}

\subsubsection*{Viscosidade} 

A viscosidade do sangue é uma medida de resistência à deformação em uma dada taxa. Alguma literatura descreve como o balanço entre a espessura e a elasticidade do fluido. Determina parâmetros importantes como a fricção sobre a parede arterial; o trabalho miocárdico; o $V\text{O}_{\mathrm{2}}$ periférico; a \emph{resistência vascular}; \emph{pré e pós carga}; \emph{perfusão}; e a \emph{\textbf{pressão arterial}}.\\

A viscosidade $\mu$ do sangue humano a 37 $^\text{o}$C é usualmente $\approx 3,5 \cdot 10^{-3} \ \ \text{Pa}\cdot{{\text{s}}^{-1}}$. Assumindo densidade $rho \approx 3,5 cP$, tem-se a viscosidade cinemática $\nu$ aproximada:

\begin{equation*}
    \nu = \cfrac{\mu}{\rho} \approx 3,5 \cdot 10^{-6} \cfrac{m^2}{s}
\end{equation*}

Os determinantes primários da viscosidade sanguínea são:

\begin{enumerate}
    \item Viscosidade do plasma
    \begin{itemize}
        \item $\implies$ $\left[\text{H}_\mathrm{2}\text{O}\right]$
        \item $\implies$ $\left[\text{DNA + RNA + PTN + mRNA + CHO + POLY + \ \dots}\right]$
    \end{itemize}
    \item Hematócrito (\% de hemácias no volume total de plasma)
    \begin{itemize}
        \item Valores de normalidade: $40,7 \to 50,3 \%$ para homens e $36,0 \to 44,3 \%$ para mulheres
        \item Traduz a capacidade que o sistema corpóreo tem de entregar O$_2$ aos tecidos
        \item É composto histoquimicamente pela hemoglobina (Hb) em g/dL
        \item Tem o \textbf{maior impacto} na viscosidade total 
        \item Para uma variação unitária no Ht $x_1 - x_0 = 1$, há um aumento na viscosidade $y = 4 \%$, para Ht $40 \ - \ 50 \%$
        \item Em valores anormais (Ht $\ge 60 \%$), $x_1 - x_0 \implies \uparrow 10 \%$
    \end{itemize}
\end{enumerate}
\subsubsection*{Propriedades mecânicas das hemácias}

Considera-se um fluido não-Newtoniano; \\

Implica na influência da taxa de cisalhamento/gradiente de velocidade do mesmo $\longrightarrow$ quanto maior o gradiente, menor a viscosidade; \\

Pico sistólico ou exercício; \\

A viscosidade aumenta com a taxa de agregação das células vermelhas (influência plaquetária);

\newpage

\begin{equation}
    \mu \sim \text{Ht} \cdot \dot \gamma \nonumber
\end{equation}

onde $\mu$ é a viscosidade e $\dot \gamma$ é o taxa de cisalhamento. A força de cisalhamento $\tau$ é dada por: \\

\begin{equation*}
    \tau = \mu_{\text{eff}} (\dot \gamma) \dot \gamma
\end{equation*}

para um fluido não-Newtoniano generalizado (não-linear), onde $\mu_{\text{eff}}$ é a viscosidade aparente. \\

\subsubsection*{O Problema do Sangue como Fluido Não-Newtoniano}

Para determinar a viscosidade do sangue, não há um modelo que governe seu comportamento, principalmente pela natureza não-linear/não-Newtoniana do sangue. Dessa maneira, a \textbf{terceira equação de Maxwell}, que é a \textbf{Lei de Faraday para Força Eletromotriz por Indução Eletromagnética} pode ser mimetizada para aproximar a \emph{a força motriz} no fluido não-Newtoniano:

\begin{align*}
    \rho \left [ \cfrac{\partial V}{\partial t} + (V \cdot \nabla)V \right] =           \nabla \cdot T + J \times \beta& \\[16pt]
    \nabla \cdot V = \nabla \cdot \beta = \nabla \times \beta = \mu_m J& \\[10pt]
    \nabla \times E + \cfrac{\partial \beta}{\partial t} = 0& \\[10pt]
    \nabla \times E = - \cfrac{\partial \beta}{\partial t}& \\[10pt]
    J = \sigma E& \\[16pt]
    \mathbf{J} \times \beta = -\sigma \beta_0^2
\end{align*} \\
onde $V$ é o módulo do vetor velocidade; $rho$ a densidade; $J$ a densidade das partículas; $\beta$ o campo vetorial total velocidade; $\mu$ a permeabilidade; $E$ o campo vetorial de tensão de cisalhamento; e $sigma$ a resistência vascular periférica. \\

\newpage
\subsubsection*{Viscosidade Modelada por GLMM}

Com o uso de modelos lineares generalizados, há um modelo que sugere a viscosidade sanguínea em função da taxa de cisalhamento por meio de uma \textbf{função de distribuição} que assume propriedades \textbf{transcendentais}:

\begin{equation*}
    \eta (\gamma^0) = \eta_{\infty} + (\eta_0 - \eta_{\infty}) {\cfrac{\left[1 + log(1 + \Lambda \gamma^0)\right]}{(1 + \Lambda \gamma^0)}
\end{equation*}

onde $\Lambda$ é uma constante temporal; $\gamma^0$ é a taxa de cisalhamento; a viscosidade aparente é $\eta$. \emph{Considerando} que a tensão de cisalhamento é a energia transferida para a \emph{parede arterial} por interação com o fluido em movimento:

\begin{equation*}
    T = \eta \times \gamma^0
\end{equation*}

onde $T$ é a tensão de cisalhamento, e a taxa de cisalhamento $\gamma$ pode ser lida como $\cfrac{\partial v (r)}{\partial r}$, associada com a velocidade de fluxo.

\subsection*{Propriedades Elásticas do Sangue}

O sangue humano é viscoelástico, implicando em interação entre energia mecânica e cinética em função do tempo. \emp{O modelo de Maxwell} para continuidade e equações de conservação se aplica: \textbf{o sangue é complexo com componentes discontínuos em seu volume. Tem tamanho e forma irregulares, também as células vermelhas não estão normalmente distribuídas pelo volume dos vasos, implicando em gradientes de velocidade e pressão}. O modelo de Maxwell tem limitações. \\

A presença de uma força $\vec{F}$, que se soma ao vetor velocidade $\vec{v}$, deforma o fluido de maneira que a parte que toca a parede interior da camada íntima exerce resistência, \emph{exigindo} da componente elástica do fluido e de tensão. A força $\vec{F}$, em condições normais, vem do \textbf{pulso cardíaco}. \\

Dessa forma, a tensão de cisalhamento é: %%% FIGURA 

\begin{equation*}
    \tau = \cfrac{F}{A}
\end{equation*}

A deformação é:

\begin{equation*}
    \gamma = \cfrac{d}{\overline h}
\end{equation*}

E o fluxo de cisalhamento é:

\begin{equation*}
    \dot \gamma = \cfrac{v}{\overline h}
\end{equation*} \\

onde, $F$ é a força; $A$ é a área; $d$ é o deslocamento; $\overline h$ é a altura; e $v$ é a velocidade.

\subsubsection*{Pulso Cardíaco Sobre o Sangue}

Para cada pulso cardíaco, há uma sístole ventricular e diástole subsequente. Na sístole, o volume de sangue por unidade de tempo e área expulsado do coração é máximo nas artérias, sendo o contrário verdadeiro para a diástole. \\ 

Desta forma, há uma deformação dada $\vec{F}_i$ em uma dada área $\text{d}A$ e num dado tempo $\text{d}t$, com comportamento \emph{senoidal} e variação de fase $phi$ entre $\tau$ e $\gamma$. Se $\phi=0$, o material é puramente elástico já que estresse e deformação \textbf{não estão defasados}. Se $\phi=90 ^\circ$, o material é puramente viscoso dada a defasagem entre estresse e deformação. \\

A variação temporal é proporcional a {$\mathbf{e^{i \omega t}}$, com $\mathbf{\omega = 2 \pi f}$}: \\[8pt]

\textbf{Shear Stress}:\\

\begin{equation*}
    \tau^{*} = \tau e^{-i \phi}
\end{equation*} \\

\textbf{Shear Strain}:\\

\begin{equation*}
    \gamma^{*} = \gamma e ^{-i \cfrac{\pi}{2}}
\end{equation*} \\

\textbf{Shear Rate}: \\

\begin{equation*}
    \dot \gamma^{*} = \dot \gamma e^{-i 0}
\end{equation*}

Então:

\begin{equation*}
    \tau^{*} = \tau ^ \prime - i \cdot \tau ^ {\prime \prime}
\end{equation*}

onde $\tau ^\prime$ é o estresse causado pela viscosidade; e $\tau ^{\prime \prime}$ o estresse elástico.

\end{document}
